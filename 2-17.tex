\documentclass[11pt]{article}
\usepackage[utf8]{inputenc}
\usepackage[english]{babel}
\usepackage{ernest}

\let\biconditional\leftrightarrow
\author{Ernest L}
\date{\today}
\title{Math 113}

\begin{document}
\maketitle
\section{Quotient Groups and Isomorphism Theorems}

    \begin{theorem}
        $Nx = Ny \biconditional xy^{-1} \in N$ for group N making right cosets
    \end{theorem}

    \begin{itemize}
        \item
        Think of normal subgroups groups as kernels of homomorphisms.
        \item 
        A subgroup is normal if and only if its left and right cosets are the same for all elements g.
    \end{itemize}

    \begin{theorem}
        $\Phi: G \to H$ is a surjective homomorphism, then $H \cong G/\ker \Phi$
    \end{theorem}
    Alternatively, the above can be expressed as:
    \begin{note}
    $\Phi: G \to H$ is a homomorphism, then $\textbf{image} (\Phi: G \to H) \cong G/\ker \Phi$
    \end{note}

    \begin{theorem}
        Let $K \triangleleft G$, $N \triangleleft G$ and $N \subseteq K \subseteq G$
        Then $K/N \triangleleft G/N$ and $(G/N)/(K/N) \cong G/K$
    \end{theorem}

    The correspondence theorem below goes in the other direction, saying that we can find a quotient group for any subgroup of of 
    the original quotient group.
    \begin{theorem}
        Suppose $T \subseteq G / N$, then there exists some subgroup $H \subseteq G$ such that $N \subseteq H$ and $H/N \cong T$
    \end{theorem}

    \begin{definition}
        A group is simple iff it contains no nontrivial proper normal subgroups.
    \end{definition}

    \begin{theorem}
        Theorem of finitely generated abelian groups: Every finitely generated abelian group is isomorphic to a direct sum of cyclic groups of the form:
        \[
            \Z_{p_1^{n_1}} \oplus \Z_{p_2^{n_2}} \oplus \cdots \oplus \Z_{p_k^{n_k}}
        \]
    \end{theorem}

\end{document}