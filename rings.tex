\documentclass[11pt]{article}
\usepackage[utf8]{inputenc}
\usepackage[english]{babel}
\usepackage{ernest}

\let\biconditional\leftrightarrow
\author{Ernest L}
\date{\today}
\title{Math 113}

\begin{document}
\maketitle
\section{Rings}

\begin{definition}
    A ring is defined under the following :
    \begin{itemize}
        \item
        Closure under addition.
        \item
        Associative addition.
        \item
        Commutative addition.
        \item
        Additive zero element
        \item
        Additive inverse element
        \item
        Closure under multiplication
        \item
        Associative multiplication  
        \item
        Multiplication is distributive over addition.
    \end{itemize}
\end{definition}

\begin{definition}
    An integral domain is a Commutative ring with identity $1_R \ne 0_R$ that satisfies: Whenever $a, b \in R$ and $ab = 0_R$, then $a = 0_R$ or $b = 0_R$.
\end{definition}

\begin{definition}
    We can classify a subset of a ring as a subring if:
    \begin{itemize}
        \item
        $S$ is closed under addition
        \item
        $S$ is closed under multiplication
        \item
        $0_R$ is in $S$
        \item
        if $a \in S$, then $-a \in S$   
    \end{itemize}
\end{definition}

\begin{definition}
    A field is a commutative ring with identity $1_R \ne 0_R$ such that every nonzero element has a multiplicative inverse. 
\end{definition}
    

\end{document}