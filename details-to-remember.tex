\documentclass[11pt]{article}
\usepackage[utf8]{inputenc}
\usepackage[english]{babel}
\usepackage{ernest}

\let\biconditional\leftrightarrow
\author{Ernest L}
\date{\today}
\title{Math 113}

\begin{document}
\maketitle
\section{Details and definitions to remember}

\begin{itemize}
    \item Endomorphism: A \underline{homomorphism} from a group to itself.
    \item Automorphism: An \underline{isomorphism} from a group to itself.
    \item An ideal is characterized by the absorbing property: $a \in I, r \in R \implies ra \in I$
    \item Normal subgroup: $gH = Hg$ for all $g \in G$
    \item Think of ideals, normal subgroups as kernels of homomorphisms.
    \begin{itemize}
        \item We can use these to take quotient groups 
        \item The first isomorphism theorem says: $G/\ker \Phi \cong \textbf{image} (\Phi: G \to H)$
    \end{itemize}
    \item An integral domain occurs iff $ab = 0 \implies a = 0$ (no zero factors) or $b = 0$ which occurss iff $ca = cb \implies a = b$ (cancellation) (these are equivalent conditions)
    \item Correspondence theorem: $N \triangleleft G$, $N \subseteq K \subseteq G$ then $K/N \triangleleft G/N$ and $(G/N)/(K/N) \cong G/K$
    \item prime ideal: $ab \in I \implies a \in I$ or $b \in I$
    \item The quotient ring: $R / I$ is an integral domain if and only if $I$ is a prime ideal.
    \begin{itemize}
        \item Proven by setting a product equal to $0$ for integral domains
    \end{itemize}
    \item The quotient ring: $R / I$ is a field if and only if $I$ is a maximal ideal.
    \begin{itemize}
        \item Proven by using the fact that $I$ is maximal ideal $\iff$ $\forall P$ such that $I \subseteq P \subseteq R$, either $P = I$ or $P = R$
    \end{itemize}
    \item Maximal ideals come from irreducible polynomials or prime numbers
    \item Burnsides Lemma: If $G$ acts on a set $X$, then the number of orbits is given by: $\frac{1}{|G|} \sum_{g \in G} |X^g|$, where $X^g$ is the set of elements fixed by $g$.    
    \item Lagrange's theorem: If $H$ is a subgroup of $G$, then $|H|$ divides $|G|$
        \begin{itemize}
            \item proven by constructing bijection between cosets, all cosets equal and partition
            \item Corrallary: For a normal subgroup $H$, the number of cosets is $|G| / |H|$
        \end{itemize}
\end{itemize}

\newpage
\section{Some results}
\begin{itemize}
    \item Every ideal of a Euclidean domain is principal:
        \begin{itemize}
            \item A Euclidean domain is an integral domain ring with an associated "order" function $N$. such that $N(0) = 0$. where every element can be divided with a unique quotient and remainder. Formally, for integral domain $I$, for all $a \in I$ and $b \in I \setminus 0$, there exists $q, r \in I$ such that $a = bq + r$ and $N(r) < N(b)$. We define $N$ as a function that represents the size. 
            \item A principal ideal is one generated by a single element
            \item If the ideal is just the zero element, then it is trivially principal
            \item Proof: Let $b \in I$ be nonzero with $N(b)$ minimal. Now, observe that we can express $a \in I$ as $a = bq + r$, such that $N(r) < N(b)$. The only way to not contradict the fact that $N(b)$ is minimal is to have $N(r) = 0$, but then this implies that $a = bq$ which is what it means to be a principal domain.
        \end{itemize}
\end{itemize}

\end{document}